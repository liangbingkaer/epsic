\documentclass[preprint]{aastex6}
%\documentclass[twocolumn]{aastex6}

\usepackage{graphicx}
%\usepackage{amsmath}
\usepackage{amssymb}
\usepackage{tikz-cd}

% bold math italic font
\newcommand{\mbf}[1]{\mbox{\boldmath $#1$}}
% bold math italic super- or subscript size font
\newcommand{\mbfs}[1]{\mbox{\scriptsize\boldmath $#1$}}

% equations
\newcommand{\Eqn}[1]{Equation~(\ref{eqn:#1})}
\newcommand{\Eqns}[3]{Equations~(\ref{eqn:#1}) #2~(\ref{eqn:#3})}
\newcommand{\eqn}[1]{equation~(\ref{eqn:#1})}
\newcommand{\eqns}[3]{equations~(\ref{eqn:#1}) #2~(\ref{eqn:#3})}

% sections
\newcommand{\Sec}[1]{Section~\ref{sec:#1}}
\newcommand{\Secs}[3]{Sections~\ref{sec:#1} #2~\ref{sec:#3}} 

% figures
\newcommand{\Fig}[1]{Figure~\ref{fig:#1}}
\newcommand{\Figs}[3]{Figures~\ref{fig:#1} #2~\ref{fig:#3}}

% tables
\newcommand{\Tab}[1]{Table~\ref{tab:#1}}
\newcommand{\Tabs}[3]{Tables~\ref{tab:#1} #2~\ref{tab:#3}}

% appendices
\newcommand{\App}[1]{Appendix~\ref{app:#1}}
\newcommand{\Apps}[1]{Appendices~\ref{app:#1}}

% symbol used for sqrt(-1)
\newcommand{\Ci}{\ensuremath{i}}

\newcommand{\C}{{\mathbb{C}}}
\newcommand{\R}{{\mathbb{R}}}

\newcommand{\rankfour}{\ensuremath{\mathbb{C}^2_{2;2}}}
\newcommand{\ranktwo}{\ensuremath{\mathbb{C}^4_{1;1}}}

\newcommand{\irow}{\mu} \newcommand{\icol}{\nu}
\newcommand{\srow}{j} \newcommand{\scol}{k}

\newcommand{\trace}{{\rm Tr}}
\newcommand{\real}{{\rm Re}}
\newcommand{\imag}{{\rm Im}}

\newcommand{\Rotation}{{\bf R}}
\newcommand{\Boost}{{\bf B}}

\newcommand{\vRotation}[1][n]{\ensuremath{\Rotation_{\mbfs{\hat #1}}}}
\newcommand{\vBoost}[1][m]{\ensuremath{\Boost_{\mbfs{\hat #1}}}}

\newcommand{\tr}[1]{\trace\ensuremath{ \left[ {#1} \right] }}
\newcommand{\re}[1]{\real\ensuremath{ \left[ {#1} \right] }}
\newcommand{\im}[1]{\imag\ensuremath{ \left[ {#1} \right] }}

\newcommand{\mean}[1]{\ensuremath{ \langle #1 \rangle }}
\newcommand{\expectation}[1]{\ensuremath{ E\left[ {#1} \right] }}
\newcommand{\smean}[1][ ]{\ensuremath{ \bar{#1} }}

\newcommand{\Linner}[2]{\ensuremath{ { {#1} \cdot {#2} } } }

\newcommand{\bilinear}[2]{\ensuremath{\left({#1},{#2}\right)}}

\newcommand{\outerBilinear}[2]{\ensuremath{{#1}\otimes{#2}}}
\newcommand{\stimes}{\ensuremath{\tilde{\otimes}}}
\newcommand{\spinorBilinear}[2]{\ensuremath{{#1}\,\stimes\,{#2}}}

% Define a bilinear operator \Delta(A,B)=(A-B)\otimes(A-B)
% \newcommand{\outerDiff}[2]{\ensuremath{{\mbf\Delta}({#1},{#2})}}

% Or, use the second tensor power
\newcommand{\outerDiff}[2]{\ensuremath{({#1}-{#2})^{\otimes 2}}}
\newcommand{\outerSymm}[2]{\ensuremath{{#1}\,\tilde{\odot}\,{#2}}}

\newcommand{\outerMueller}[2]{\ensuremath{{\bf M}_\otimes\bilinear{#1}{#2}}}
\newcommand{\spinorMueller}[2]{\ensuremath{{\bf M}_{\stimes}\bilinear{#1}{#2}}}

\newcommand{\exterior}[2]{\ensuremath{{#1}\wedge{#2}}}
\newcommand{\dc}[2]{\ensuremath{{#1}\,\mbf{:}\,{#2}}}

\newcommand{\rotat}{\ensuremath{\vRotation(\phi)}}
\newcommand{\boost}{\ensuremath{\vBoost(\beta)}}

\newcommand{\pauli}[1]{\ensuremath{\mbf{\sigma}_{#1}}}

\newcommand{\inv}[1]{\ensuremath{ {#1}^2 }}
\newcommand{\norm}[1]{\ensuremath{ \|{#1}\| }}

\newcommand{\element}[3]{\ensuremath{ \left\{ {#1} \right\}_{#2}^{#3} }}
\newcommand{\Celement}[1]{\element{#1}{\irow}{\icol}}

\newcommand{\instI}{\ensuremath{\xi}}

\shorttitle   {Disjoint, Superposed, and Composite Samples}
\shortauthors {van Straten \& Tiburzi}

\received{}

\begin{document}

\title{ {\sc epsic}: Electromagnetic Polarization Simulation in C++ }

\author{W. van Straten}
\affil{Institute for Radio Astronomy \& Space Research,
Auckland University of Technology,
Private Bag 92006,
Auckland 1142,
New Zealand}
\email{willem.van.straten@aut.ac.nz}

\begin{abstract}

  {\sc epsic} can be used to simulate the effects of integration over finite samples when more than one source is present. The sources can be disjoint (mutually exclusive), such that only one source emits at a given instant, or superposed, such that the electric fields are summed. It is also possible to simulate integration over a composite sample of unresolved disjoint modes.

\end{abstract}


%%%%%%%%%%%%%%%%%%%%%%%%%%%%%%%%%%%%%%%%%%%%%%%%%%%%%%%%%%%%%%%%%%%%%%
%%%%%%%%%%%%%%%%%%%%%%%%%%%%%%%%%%%%%%%%%%%%%%%%%%%%%%%%%%%%%%%%%%%%%%
%%%%%%%%%%%%%%%%%%%%%%%%%%%%%%%%%%%%%%%%%%%%%%%%%%%%%%%%%%%%%%%%%%%%%%

\section {Introduction}

{\sc epsic} performs the following Monte Carlo simulation.

\begin{enumerate}
\item Generate a sequence of $M$ random electric field vector
  instances $\mbf{e}$, each with statistically independent and
  identically distributed (iid) circular complex normal components;
  such a sequence is described by the population mean Stokes
  parameters [1,0,0,0].
  
\item To yield the desired population mean Stokes parameters, $S_\irow$,
  transform each electric field vector instance by the Hermitian square
  root of $2\mbf\rho=S_\irow\,\pauli{\irow}$.

\item Optionally perform amplitude modulation by multiplying each
  instance of $\mbf{e}$ by an iid random variate $u$ that is drawn
  from a log-normal distribution.  The log-normally distributed
  variate is generated from a normally distributed iid variate with
  zero mean and standard deviation $\varsigma$ and is normalized by
  the mean of the distribution, $\langle u \rangle =
  \exp(\varsigma^2/2)$, such that the mean of the amplitude modulating
  function is unity.
  %
  % CHANGED: The simulation was updated to support appendix A
  %
  To simulate square subpulses defined by the sub-sample size $n$, as
  described in \App{gxv+99}, a single value of $u$ is applied to $n$
  consecutive instances of $\mbf{e}$.

\item If simulating \emph{superposed samples}, repeat all of the previous
  steps to produce $M$ instances of the electric field vector in the
  other mode then, for each instance of the electric field vectors
  from modes $A$ and $B$, produce $M$ new instances
  $\mbf{e}=\mbf{e}_A+\mbf{e}_B$.
  
\item Compute the instantaneous Stokes parameters,
  $s_\irow=\mbf{e}^\dagger\pauli{\irow}\mbf{e}$.

\item Optionally divide the sequence of $M$ instantaneous Stokes
  vectors into mutually exclusive Stokes samples of $n$ instances,
  yielding a sequence of $N=M/n$ Stokes samples.  This step is not
  optional when simulating composite samples.

\item If simulating \emph{composite samples}, replace $(1-f)n$ instances
  in each Stokes sample with instantaneous Stokes vectors in the
  other mode.

\item If simulating \emph{disjoint samples}, replace $(1-F)N$ Stokes
  samples with Stokes samples that contain only instantaneous
  Stokes vectors in the other mode.

\item For each Stokes sample, compute the sample mean Stokes
  parameters $\bar{S}_\irow$.

\item Compute the $4\times4$ covariances between the Stokes parameters
  using either the $M$ instantaneous Stokes parameters or the $N$
  sample mean Stokes parameters.  Verify that the computed covariance
  matrix matches the theoretical prediction within the uncertainty due
  to noise.
  
\end{enumerate}
%


\bibliographystyle{aasjournal}
\bibliography{journals,modrefs,psrrefs,../local,crossrefs}
\end{document}


