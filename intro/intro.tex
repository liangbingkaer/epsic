%\documentclass[preprint]{aastex6}
\documentclass[twocolumn]{aastex6}

\usepackage{graphicx}
\usepackage{amsmath}
\usepackage{amssymb}

% bold math italic font
\newcommand{\mbf}[1]{\mbox{\boldmath $#1$}}
% bold math italic super- or subscript size font
\newcommand{\mbfs}[1]{\mbox{\scriptsize\boldmath $#1$}}

% equations
\newcommand{\Eqn}[1]{Equation~(\ref{eqn:#1})}
\newcommand{\Eqns}[3]{Equations~(\ref{eqn:#1}) #2~(\ref{eqn:#3})}
\newcommand{\eqn}[1]{equation~(\ref{eqn:#1})}
\newcommand{\eqns}[3]{equations~(\ref{eqn:#1}) #2~(\ref{eqn:#3})}

% sections
\newcommand{\Sec}[1]{Section~\ref{sec:#1}}
\newcommand{\Secs}[3]{Sections~\ref{sec:#1} #2~\ref{sec:#3}} 

% figures
\newcommand{\Fig}[1]{Figure~\ref{fig:#1}}
\newcommand{\Figs}[3]{Figures~\ref{fig:#1} #2~\ref{fig:#3}}

% tables
\newcommand{\Tab}[1]{Table~\ref{tab:#1}}
\newcommand{\Tabs}[3]{Tables~\ref{tab:#1} #2~\ref{tab:#3}}

% appendices
\newcommand{\App}[1]{Appendix~\ref{app:#1}}
\newcommand{\Apps}[1]{Appendices~\ref{app:#1}}

% symbol used for sqrt(-1)
\newcommand{\Ci}{\ensuremath{i}}

\newcommand{\C}{{\mathbb{C}}}
\newcommand{\R}{{\mathbb{R}}}

\newcommand{\rankfour}{\ensuremath{\mathbb{C}^2_{2;2}}}
\newcommand{\ranktwo}{\ensuremath{\mathbb{C}^4_{1;1}}}

\newcommand{\irow}{\mu} \newcommand{\icol}{\nu}
\newcommand{\jrow}{\kappa} \newcommand{\jcol}{\lambda}
\newcommand{\srow}{j} \newcommand{\scol}{k}

\newcommand{\trace}{{\rm Tr}}
\newcommand{\real}{{\rm Re}}
\newcommand{\imag}{{\rm Im}}

\newcommand{\Rotation}{{\bf R}}
\newcommand{\Boost}{{\bf B}}

\newcommand{\vRotation}[1][n]{\ensuremath{\Rotation_{\mbfs{\hat #1}}}}
\newcommand{\vBoost}[1][m]{\ensuremath{\Boost_{\mbfs{\hat #1}}}}

\newcommand{\tr}[1]{\trace\ensuremath{ \left[ {#1} \right] }}
\newcommand{\re}[1]{\real\ensuremath{ \left[ {#1} \right] }}
\newcommand{\im}[1]{\imag\ensuremath{ \left[ {#1} \right] }}

\newcommand{\sinc}{\mathrm sinc}
\newcommand{\bw}{\ensuremath{ \Delta\nu }}

\newcommand{\mean}[1]{\ensuremath{ \langle #1 \rangle }}
\newcommand{\expectation}[1]{\ensuremath{ E\left[ {#1} \right] }}
\newcommand{\smean}[1][ ]{\ensuremath{ \bar{#1} }}

\newcommand{\Linner}[2]{\ensuremath{ { {#1} \cdot {#2} } } }

\newcommand{\bilinear}[2]{\ensuremath{\left({#1},{#2}\right)}}

\newcommand{\outerBilinear}[2]{\ensuremath{{#1}\otimes{#2}}}
\newcommand{\stimes}{\ensuremath{\tilde{\otimes}}}
\newcommand{\spinorBilinear}[2]{\ensuremath{{#1}\,\stimes\,{#2}}}

% the second tensor power of a difference
\newcommand{\outerDiff}[2]{\ensuremath{({#1}-{#2})^{\otimes 2}}}
\newcommand{\outerSymm}[2]{\ensuremath{{#1}\,\tilde{\odot}\,{#2}}}

\newcommand{\outerMueller}[2]{\ensuremath{{\bf M}_\otimes\bilinear{#1}{#2}}}
\newcommand{\spinorMueller}[2]{\ensuremath{{\bf M}_{\stimes}\bilinear{#1}{#2}}}

\newcommand{\exterior}[2]{\ensuremath{{#1}\wedge{#2}}}
\newcommand{\dc}[2]{\ensuremath{{#1}\,\mbf{:}\,{#2}}}

\newcommand{\rotat}{\ensuremath{\vRotation(\phi)}}
\newcommand{\boost}{\ensuremath{\vBoost(\beta)}}

\newcommand{\pauli}[1]{\ensuremath{\mbf{\sigma}_{#1}}}

\newcommand{\inv}[1]{\ensuremath{ {#1}^2 }}
\newcommand{\norm}[1]{\ensuremath{ \|{#1}\| }}

\newcommand{\element}[3]{\ensuremath{ \left\{ {#1} \right\}_{#2}^{#3} }}
\newcommand{\Celement}[1]{\element{#1}{\irow}{\icol}}

\newcommand{\instI}{\ensuremath{\xi}}

\shorttitle   {Introduction to Polarimetry}
\shortauthors {van Straten}


\begin{document}

\title{ Introduction to Radio Astronomical Polarimetry }

\author{W. van Straten}
\affil{Institute for Radio Astronomy \& Space Research,
Auckland University of Technology,
Private Bag 92006,
Auckland 1142,
New Zealand}
\email{willem.van.straten@aut.ac.nz}

\begin{abstract}

  This paper presents an introduction to the concepts and mathematical
  foundations of polarimetry.
  
\end{abstract}


%%%%%%%%%%%%%%%%%%%%%%%%%%%%%%%%%%%%%%%%%%%%%%%%%%%%%%%%%%%%%%%%%%%%%%
%%%%%%%%%%%%%%%%%%%%%%%%%%%%%%%%%%%%%%%%%%%%%%%%%%%%%%%%%%%%%%%%%%%%%%
%%%%%%%%%%%%%%%%%%%%%%%%%%%%%%%%%%%%%%%%%%%%%%%%%%%%%%%%%%%%%%%%%%%%%%

%%%%%%%%%%%%%%%%%%%%%%%%%%%%%%%%%%%%%%%%%%%%
%%%%%%%%%%%%%%%%%%%%%%%%%%%%%%%%%%%%%%%%%%%%
\section{Polarization}
\label{sec:polarization}

For a plane-propagating transverse electromagnetic wave, there exist
two independent solutions to Maxwell's equations, representing two
orthogonal senses of polarization.  Radio receiver systems must
differentiate between these two senses in order to fully describe the
vector state of the observed radiation.  A dual-polarization receiver
is therefore designed with two receptors, or probes, that are ideally
sensitive to orthogonal polarizations.  Define the transverse electric
field vector,
\begin{equation}
\mbf{e}(t) =
 \left( 
    \begin{array}{c}
      e_0(t) \\
      e_1(t)
    \end{array}
  \right)
\end{equation}
where $e_0(t)$ and $e_1(t)$ are the complex-valued analytic signals
associated with two real time series, providing the instantaneous
amplitudes and phases of the two orthogonal senses of polarization
(see \App{analytic} for more details).

The polarization of electromagnetic radiation is described by the
second-order statistics of $\mbf{e}$, as represented by the complex
$2\times2$ coherency matrix \citep{bw80}
\begin{equation}
  \label{eqn:coherency}
  \mbf{\rho}\equiv\langle\mbf{e \otimes e}^\dagger\rangle
  =\left( 
    \begin{array}{cc}
      \langle e_0e_0^*\rangle & \langle e_0e_1^*\rangle \\
      \langle e_1e_0^*\rangle & \langle e_1e_1^*\rangle
    \end{array}
  \right).
\end{equation}
Here, the angular brackets denote an ensemble average, $\mbf\otimes$
is the tensor product, and $\mbf{e}^\dagger$ is the Hermitian transpose
of $\mbf{e}$.

The coherency matrix is self-adjoint, or Hermitian,
i.e.\ $\mbf{\rho}=\mbf{\rho}^\dagger$, and can be written as a linear
combination of four Hermitian basis matrices,
%
\begin{equation}
  {\mbf\rho} = S_\irow\,\pauli{\irow} / 2,    \label{eqn:combination}
\end{equation}
where $S_\irow$ are the four Stokes parameters, Einstein
notation is used to imply a sum over repeated indeces,
$0\le\irow\le3$, $\pauli{0}$ is the $2\times2$ identity matrix, and
\begin{eqnarray}
{\mbf{\sigma}_1} = \left( \begin{array}{cc}
1 & 0 \\
0 & -1 
\end{array}\right)
&
\mbf{\sigma}_2 = \left( \begin{array}{cc}
0 & 1 \\
1 & 0 
\end{array}\right)
& 
\mbf{\sigma}_3 = \left( \begin{array}{cc}
0 & -\Ci \\
\Ci & 0
\end{array}\right)
\label{pauli}
\end{eqnarray}
%
are the Pauli matrices.  The Pauli matrices are traceless
(i.e.\ $\tr{\pauli{i}}=0$, where $\trace$ is the matrix trace
operator) and satisfy
\begin{equation}
\pauli{i}\,\pauli{j} = \delta_{ij}\,\pauli{0} + i\epsilon_{ijk}\,\pauli{k},
\end{equation}
where $\epsilon_{ijk}$ is the permutation symbol and summation over
the index $k$ is implied.  Using these properties, it is easily shown that
%
\begin{equation}
S_\irow = \trace(\pauli{\irow}\mbf{\rho}).  \label{eqn:trace}
\end{equation}
%
Equivalent expressions of the Stokes parameters are given by
%
\begin{equation}
S_\irow = \langle \mbf{e}^\dagger \pauli{\irow} \mbf{e} \rangle \label{eqn:inner}
\end{equation}
and
\begin{equation}
S_\irow = \dc{\pauli{\irow}}{\mbf\rho}     \label{eqn:projection}
\end{equation}
%
where the $\mbf{:}$ operator represents tensor double contraction, a
tensor product followed by contraction over two pairs of indeces.  The
double contraction of two matrices {\bf A} and {\bf B} yields a scalar
quantity defined by
%
\begin{equation}
\dc{\bf A}{\bf B} \equiv A_\mu^\nu B_\nu^\mu.
\label{eqn:double_contraction_matrices}
\end{equation}
%
\Eqn{combination} expresses the coherency matrix as a linear
combination of Hermitian basis matrices; \Eqn{projection} represents
the Stokes parameters as the projections of the coherency matrix onto
the basis matrices.  Because $\mbf\rho$ is Hermitian, the Stokes
parameters are real-valued.  
%

Any one of \Eqns{trace}{through}{projection} can be used to
derive the following expressions for the Stokes parameters 
\begin{eqnarray}
 S_0=&\langle|e_0(t)|^2\rangle+\langle|e_1(t)|^2\rangle\label{eqn:stokesI} \\
 S_1=&\langle|e_0(t)|^2\rangle-\langle|e_1(t)|^2\rangle\label{eqn:stokesQ} \\
 S_2=&2\re{\langle e_0^*(t)e_1(t)\rangle}\label{eqn:stokesU} \\
 S_3=&2\im{\langle e_0^*(t)e_1(t)\rangle}.\label{eqn:stokesV}
\end{eqnarray}

It proves useful to organize the four real-valued 
Stokes parameters into scalar and vector components,
[$S_0,\mbf{S}$], where $S_0$ is the total intensity and
$\mbf{S}=(S_1,S_2,S_3)$ is the polarization vector.

Given a Cartesian basis in which the radiation propagates in the
direction of the $\mbf{\hat z}$ axis, and the electric field is
measured by its projection along the $\mbf{\hat x}$ and $\mbf{\hat y}$
axes,
\begin{equation}
\mbf{e}(t) =
 \left( 
    \begin{array}{c}
      e_x(t) \\
      e_y(t)
    \end{array}
  \right)
\end{equation}
and $\mbf{S}=(Q,U,V)$.
%
Some receivers employ waveguide structures (for example, a quarter
waveplate) to convert from linear to circular polarization.
In a basis defined in terms of orthogonal senses of circular
polarization \citep[e.g.\ see][for more details]{vmjr10},
\begin{equation}
\mbf{e}(t) =
 \left( 
    \begin{array}{c}
      e_l(t) \\
      e_r(t)
    \end{array}
  \right)
\end{equation}
%
and $\mbf{S}=(V,Q,U)$. To foster an intuitive understanding of the
mathematical definitions of the Stokes parameters, consider some
special cases of polarization state.

\subsection{ 100\% Polarized Radiation }

Six special cases of fully polarized radiation are considered in the
Cartesian basis; they are organized into the following three groups:

\begin{enumerate}
\item Stokes Q: $e_x=0$ or $e_y=0$
\item Stokes U: $e_x=e_y$ or $e_x=-e_y$
\item Stokes V: $e_x=-i e_y$ or $e_x=i e_y$
\end{enumerate}

\subsubsection {Stokes Q}

If $e_y=0$, then the radiation is 100\% linearly polarized with
the electric field vector oscillating only along the x-axis.  In this
case, the total intensity $S_0$ is equal to the variance of $e_x$,
$S_1=S_0$ and $S_2=S_3=0$.  This corresponds to positive Stokes $Q$.

If $e_x=0$, then the radiation is 100\% linearly polarized with
electric field vector oscillating only along the y-axis.  In this
case, the total intensity $S_0$ is equal to the variance of $e_y$,
$S_1=-S_0$ and $S_2=S_3=0$.  This corresponds to negative Stokes $Q$.

\subsubsection {Stokes U}

If $e_x=e_y$, then the radiation is 100\% linearly polarized with
the electric field vector oscillating only along an axis that is rotated
by 45 degrees with respect to the x-axis.  In this case, the variances
of $e_x$ and $e_y$ are equal, the total intensity $S_0$ is equal to
twice the variance of $e_x$, $S_1=0$, $S_2=S_0$, and $S_3=0$.  This
corresponds to positive Stokes $U$.

If $e_x=-e_y$, then the radiation is 100\% linearly polarized with
the electric field vector oscillating only along an axis that is rotated
by -45 degrees with respect to the x-axis.  In this case, the variances
of $e_x$ and $e_y$ are equal, the total intensity $S_0$ is equal to
twice the variance of $e_x$, $S_1=0$, $S_2=-S_0$, and $S_3=0$.  This
corresponds to negative Stokes $U$.

\subsubsection {Stokes V}

If $e_x=-i e_y$, then the phase of $e_y$ leads that of $e_x$ by 90
degrees and the radiation is 100\% circularly polarized with the electric
field vector tracing a counter-clockwise circle in the x-y plane; this
is defined by the IEEE as left-hand circularly polarized (LCP).  In
this case, the variances of $e_x$ and $e_y$ are equal, the total
intensity $S_0$ is equal to twice the variance of $e_x$, $S_1=0$,
$S_2=0$, and $S_3=S_0$.  This corresponds to positive Stokes $V$,
which is contrary to the IAU convention \citep[see][for more
  details]{vmjr10}.

If $e_x=i e_y$, then the phase of $e_x$ leads that of $e_y$ by 90
degrees and the radiation is 100\% circularly polarized with the electric
field vector tracing a clockwise circle in the x-y plane; this is
defined by the IEEE as right-hand circularly polarized (RCP).  In this
case, the variances of $e_x$ and $e_y$ are equal, the total intensity
$S_0$ is equal to twice the variance of $e_x$, $S_1=0$, $S_2=0$, and
$S_3=-S_0$.  This corresponds to negative Stokes $V$.

\subsection { Unpolarized Radiation }

It is easy to see in the case of Stokes Q that polarized radiation can
can be seen as an incoherent superposition of orthogonally polarized
(and 100\% polarized) modes.  Incoherent means that the modes are uncorrelated
(i.e.\ $S_2=S_3=0$) and not necessarily statistically independent.
When only one mode is present, $S_1 = \pm S_0$ and the signal is 100\%
polarized; when incoherent modes are present with equal power, $S_1=0$
and the signal is unpolarized.

In fact, any polarized state can be represented as an incoherent sum
of orthogonally polarized states.  This can be seen by expressing the
coherency matrix as a similarity transformation known as its eigen
decomposition,
%
\begin{equation}
\mbf{\rho}={\bf{R}}\left( \begin{array}{cc}
\lambda_0 & 0 \\
0 & \lambda_1
\end{array}\right){\bf{R}}^{-1}.
\label{eqn:eigen}
\end{equation}
Here, ${\bf{R}}=(\mbf{e}_0\,\mbf{e}_1)$ is a $2\times2$ matrix with
columns equal to the eigenvectors of $\mbf{\rho}$, and $\lambda_m$ are
the corresponding eigenvalues, given by
$\lambda=(S_0\pm|\mbf{S}|)/2=(1 \pm P)S_0/2$, where $P=|\mbf{S}|/S_0$
is the degree of polarization ($0\le P\le1$).  If the signal is
completely polarized, then $\lambda_1=0$ and the degree of
polarization $P=1$.  If the signal is unpolarized, then there is a
single 2-fold degenerate eigenvalue, $\lambda=S_0/2$, the degree of
polarization $P=0$, and ${\bf R}$ is undefined; that is, an
unpolarized signal is unpolarized in any basis.

If the eigenvectors are normalized such that
${\mbf{e}_k}^\dagger\mbf{e}_k=1$, then \eqn{eigen} is equivalent to a
congruence transformation by a unitary matrix (${\bf R}^\dagger={\bf
  R}^{-1}$).
%
In the natural basis defined by ${\bf{R}}^\dagger$, the eigenvalues
$\lambda_m$ are equal to the variances of two uncorrelated signals
received by orthogonally polarized receptors described by the
eigenvectors.
%
The total intensity, $S_0=\lambda_0+\lambda_1$; the
polarized intensity, $S_1=|\mbf{S}|=\lambda_0-\lambda_1$; and
$S_2=S_3=0$.  
%
That is, ${\bf{R}}^\dagger$ rotates the basis such that the mean
polarization vector points along $S_1$.

%
% The Stokes parameters may be mapped onto a point in the Poincare
% sphere by $\mbf{p}=\mbf{S}/S_0$


%%%%%%%%%%%%%%%%%%%%%%%%%%%%%%%%%%%%%%%%%%%%
%%%%%%%%%%%%%%%%%%%%%%%%%%%%%%%%%%%%%%%%%%%%

\section{Linear Transformations}
\label{sec:transformations}

In the narrow-band (or quasi-monochromatic) approximation of an
electromagnetic wave (see \App{linear_transformations}), linear
transformations of the electric field vector are represented by
%
\begin{equation}
\mbf{e}^\prime(t)={\bf J}\mbf{e}(t),
\label{eqn:field_transformation}
\end{equation}
where {\bf J} is a $2\times2$ complex-valued Jones matrix.

Substitution of
$\mbf{e}^\prime={\bf J}\mbf{e}$ into the definition of the coherency
matrix yields the congruence transformation,
%
\begin{equation}
{\mbf{\rho}^\prime}={\bf{J}}\mbf{\rho}{\bf{J}}^\dagger,
\label{eqn:congruence}
\end{equation}
%%
%% REF 4.
%%
which forms the basis of the various coordinate transformations that
are exploited throughout this work.
%
If ${\bf{J}}$ is non-singular, it can be decomposed into the product
of a Hermitian matrix and a unitary matrix known as its polar
decomposition,
\begin{equation}
\label{eqn:polar}
{\bf J} = J \, \boost \, \rotat,
\end{equation}
where $J=|{\bf J}|^{1/2}$, $|{\bf J}|$ is the determinant of {\bf J},
\boost\ is positive-definite Hermitian,
%% \[
%% \left[\boost\right]^\dagger=\boost;
%% \]
and \rotat\ is unitary.
%% \[
%% \left[\rotat\right]^\dagger=\left[\rotat\right]^{-1}.
%% \]
%
Under the congruence transformation of the coherency matrix, the
Hermitian matrix
\begin{equation}
\label{eqn:Boost}
\boost = \pauli{0}\cosh\beta + \mbf{\hat{m}\cdot\sigma}\sinh\beta
\end{equation}
effects a Lorentz boost of the Stokes four-vector along the $\mbf{\hat
  m}$ axis by a hyperbolic angle $2\beta$.  In the above equation,
$\mbf{\sigma}$ is a 3-vector whose components are the Pauli spin
matrices.
%%
%% REF 4.
%%
As the Lorentz transformation of a spacetime event mixes temporal and
spatial dimensions, the polarimetric boost mixes total and polarized
intensities, thereby altering the degree of polarization.
%
In contrast, the unitary matrix
\begin{equation}
\label{eqn:Rotation}
\rotat = \pauli{0}\cos\phi + \Ci\mbf{\hat{n}\cdot\sigma}\sin\phi
\end{equation}
rotates the Stokes polarization vector about the $\mbf{\hat n}$ axis by
an angle $2\phi$.
%%
%% REF 4.
%%
As the orthogonal transformation of a vector in Euclidean space
preserves its length, the polarimetric rotation leaves the degree of
polarization unchanged.
%

These geometric interpretations promote a more intuitive treatment of
the matrix equations that typically arise in polarimetry.
%
Boost transformations can be utilized to convert unpolarized radiation
into partially polarized radiation, and rotation transformations can
be used to choose the orthonormal basis that maximizes symmetry.

Both \boost\ and \rotat\ are unimodular (i.e.\ $|\boost|=1$ and
$|\rotat|=1$); therefore, because $|{\bf AB}|=|{\bf A}||{\bf B}|$,
congruence transformation of the coherency matrix by either \boost\ or
\rotat\ preserves the determinant.  The determinant of the coherency
matrix is therefore an invariant of boost and rotation transformations
(but not of scalar multiplication).  Using \Eqn{combination} it is
easy to show that the Lorentz invariant of a Stokes four-vector is
equal to four times the determinant of the coherency matrix; that is,
\begin{equation}
\inv{S} \equiv S_0^2 - |\mbf{S}|^2 = 4 |\mbf{\rho}|
\end{equation}
As with the spacetime null interval, no linear transformation of the
electric field can alter the degree of polarization of a completely
polarized source.

Each axis-angle parameterization of \boost\ and \rotat\ has three free
parameters: a unit vector that defines the axis of symmetry of the
transformation and an angle.  Combined with the real and imaginary
parts of the complex-valued $J$, \Eqn{polar} has eight degrees of
freedom, as expected for a complex-valued $2\times2$ matrix.  However,
the coherency matrix is insensitive to the absolute phase of $J$
because $J$ (on the left) is multiplied by $J^*$ (on the right) in
\Eqn{coherency}.  Therefore, only seven degrees of freedom matter
in single-dish polarimetry and $J$ can be replaced by the real-valued
absolute gain $G$.

Using \Eqns{combination}{and}{trace}, the congruence
transformation of the coherency matrix can be expressed as an
equivalent linear transformation of the associated Stokes parameters
by a real-valued $4\times4$ Mueller matrix {\bf M}, as defined by
%
\begin{equation}
 S_\irow^\prime = M_\irow^\icol S_\icol
\label{eqn:linear_to_Mueller}
\end{equation}
%
where
\begin{equation}
 M_\irow^\icol = \frac{1}{2} \tr{\pauli{\irow}{\bf J}\pauli{\icol}{\bf J}^\dagger}.
\end{equation}
%
Although there is an unique Mueller matrix for every Jones matrix, the
converse is not true.  Mueller matrices that do not have an equivalent Jones
matrix are known as ``impure''.

\begin{appendix}

%%%%%%%%%%%%%%%%%%%%%%%%%%%%%%%%%%%%%%%%%%%%
%%%%%%%%%%%%%%%%%%%%%%%%%%%%%%%%%%%%%%%%%%%%
\section{The Analytic Signal}
\label{app:analytic}

The voltage signal from each receptor is a real-valued function of
time, or process, that may be represented by its associated analytic
signal.

The analytic signal, also known as Gabor's complex signal, is a
complex-valued representation of a real-valued process that provides
its instantaneous amplitude and phase.  In order to define the
analytic signal associated with a process, $x(t)$, it is first
necessary to define the Hilbert transform \citep{pap65},
\begin{equation}
  \hat x(t) \equiv
	{1\over\pi} \int_{-\infty}^\infty {x(\tau)\over t-\tau} \,d\tau.
\label{eqn:hilbert}
\end{equation}
The discontinuity in $\hat x(t)$ at $t=\tau$ is avoided by taking the
Cauchy principal value.
%, ${\mathcal{P}}(\tau)$:
%\[
%  {\mathcal P}(\tau) \equiv \lim_{\epsilon\to 0} 
%	\left[{\int^{t-\epsilon}_{-\infty} {x(\tau)\over\tau-t}\,d\tau
%	+ \int^\infty_{t+\epsilon} {x(\tau)\over\tau-t}\,d\tau}\right]
%\]
The analytic signal associated with $x(t)$ is then defined by
\begin{equation}
  z(t) = x(t) + \Ci\hat x(t).
\end{equation}

As it is derived from the real-valued process, the analytic signal
does not contain any additional information.  However, the analytic
signals associated with two orthogonal senses of polarization, $e_0(t)$ and
$e_1(t)$, permit calculation of the coherency matrix.  The analytic signal
therefore proves to be a useful representation in radio polarimetric
studies and in the theoretical description of quadrature
down-conversion (see \App{dsb}).

%%%%%%%%%%%%%%%%%%%%%%%%%%%%%%%%%%%%%%%%%%%%
%%%%%%%%%%%%%%%%%%%%%%%%%%%%%%%%%%%%%%%%%%%%
\subsection{The Quadrature Filter}
\label{app:quadrature_filter}

The Hilbert transformation of \Eqn{hilbert} may also be
written as the convolution, $\hat{x}(t)=h(t)*x(t)$, where
\[
h(t) = {1\over{\pi t}},
\]
and the $*$ symbol is used to represent the convolution operation,
\begin{equation}
h(t)*x(t)\equiv\int_{-\infty}^{\infty}{x(\tau)h(t-\tau)d\tau}.
\end{equation}
By the convolution theorem, this transformation is equivalent to 
$\hat{X}(\nu)=H(\nu)X(\nu)$, where
\begin{equation} H(\nu) = \left\{ \begin{array}{cc}
                -\Ci  & \nu > 0 \\
                \Ci & \nu < 0
                \end{array} \right.
\end{equation}
is the Fourier transform of $h(t)$, known as the quadrature
filter \citep{pap65}.  Referring to the other commonly used functions
and their Fourier transforms in \Tab{fourier}, it is trivial to show
that the Hilbert transform of $\cos(\nu_0t)$ is equal to
$\sin(\nu_0t)$, enabling the quadrature filter to be understood as a
90\degr\ phase shifter.  Using the quadrature filter, it can also be
shown that the Fourier transform of the analytic signal, $Z(\nu)$, is
equal to zero for $\nu$ less than zero:
\begin{eqnarray}
 Z(\nu) & = & X(\nu) + \Ci\hat X(\nu) \nonumber \\
	& = & X(\nu) + \Ci H(\nu)X(\nu) \nonumber \\
	& = & \left \{ \begin{array}{cc} 
	2X(\nu)  & \nu > 0 \\
	0 & \nu < 0 
	\end{array} \right. \nonumber
\end{eqnarray}
Conversely, the analytic signal associated with $x(t)$ may be produced
by suppression of the negative frequencies in $X(\nu)$.

\begin{table}
\begin{center}
\begin{tabular}{c|c}
\hline
\hline
$x(t)$ & $X(\nu)$ \\
\hline \\ [-2mm]
$\cos(2\pi\nu_0t)$ & $\frac{1}{2}(\delta(\nu+\nu_0)+\delta(\nu-\nu_0))$ \\ [4mm]
$\sin(2\pi\nu_0t)$ & $\frac{\Ci}{2}(\delta(\nu+\nu_0)-\delta(\nu-\nu_0))$ \\ [4mm]
$h(t)=(\pi t)^{-1}$ & $H(\nu) =\left\{ \begin{array}{cc}
	       	        	-\Ci & \nu > 0 \\
	                	\Ci  & \nu < 0 \end{array} \right.$ \\ [4mm]
$\pi(t)=\sinc(\pi\bw t)$ & $\Pi(\nu/\bw) = \left\{ \begin{array}{cc}
                		0  & |\nu/\bw| > 1/2 \\
				1/2& |\nu/\bw| = 1/2 \\
               			1  & |\nu/\bw| < 1/2 \end{array} \right.$ \\ [8mm]
\hline
\end{tabular}
\end{center}
\caption
{Useful Fourier Transform pairs. The left column lists functions of
time.  In the right column, the corresponding Fourier transform is
given as a function of oscillation frequency, $\nu$. The filters,
$H(\nu)$ and $\Pi(\nu)$, are known as the quadrature and rectangle
functions, respectively.}
\label{tab:fourier}
\end{table}

%%%%%%%%%%%%%%%%%%%%%%%%%%%%%%%%%%%%%%%%%%%%
%%%%%%%%%%%%%%%%%%%%%%%%%%%%%%%%%%%%%%%%%%%%

\section{Narrow-band approximation to polarimetric transformations}
\label{app:linear_transformations}

Regardless of feed design, the electric field component of the
radiation impinging on the receiver feed horn induces a voltage in
each of the receptors and these voltages are propagated through
separate signal paths.
%
Each voltage signal therefore experiences a different series of
amplification, attenuation, mixing, and filtering before sampling or
detection is performed.  Whereas efforts are made to match the
components of the observatory equipment, each will realistically have
a unique frequency response to the input signal.  Even a simple
mismatch in signal path length will result in a relative phase
difference between the two polarizations that varies linearly with
frequency.

In fact, any physically realizable system will transform the radiation
in a manner that depends on frequency.  Where variations across the
smallest bandwidth available may be considered negligible,
post-detection calibration and correction techniques may be used to
invert the transformation and recover the original polarimetric state.
However, the transformation may vary significantly across the band,
causing the polarization vector to combine destructively when
integrated in frequency.  This phenomenon is known as ``bandwidth
depolarization'' of the signal, and results in irreversible decimation
of the degree of polarization.  It is therefore desirable to perform
polarimetric corrections at sufficiently high spectral resolution.

Consider a linear system with impulse response, $j(t)$.  Presented
with an input signal, $e(t)$, the output of this system is given by
the convolution, $e^\prime(t)=j(t)*e(t)$.  In the two-dimensional
case, each output signal is given by a linear combination of the input
signals,
\begin{eqnarray}
e_1^\prime(t)=j_{11}(t)*e_1(t) + j_{12}(t)*e_2(t), \\
e_2^\prime(t)=j_{21}(t)*e_1(t) + j_{22}(t)*e_2(t).
\end{eqnarray}

  By
defining the analytic vector, $\mbf{e}(t)$, with elements $e_1(t)$ and
$e_2(t)$, and the $2\times2$ impulse response matrix, ${\bf{j}}(t)$,
with elements $j_{mn}(t)$, we may express the propagation of a
transverse electromagnetic wave by the matrix equation,
\begin{equation}
{\mbf{e^\prime}}(t)={\bf{j}}(t)*\mbf{e}(t).
\label{eqn:convolution_t}
\end{equation}
By the convolution theorem, \Eqn{convolution_t} is equivalent to
\begin{equation}
\mbf{E^\prime}(\nu)={\bf J}(\nu)\mbf{E}(\nu),
\label{eqn:convolution_nu}
\end{equation}
where $\bf{J}(\nu)$ is the frequency response matrix with elements
$J_{mn}(\nu)$, and $\mbf{E}(\nu)$ is the vector spectrum.  In the case
of monochromatic light, or under the assumption that $\bf{J}(\nu)$ is
constant over all frequencies, matrix convolution reduces to simple
matrix multiplication in the time domain, as traditionally represented
using the Jones matrix.  However, because these conditions are not
physically realizable, the Jones matrix finds its most meaningful
interpretation in the frequency domain.

The average auto- and cross-power spectra are summarized by the
average power spectrum matrix, defined by the vector direct product,
\mbox{$\bar{\bf{P}}(\nu)
=\langle\mbf{E}(\nu)\boldsymbol{\otimes}\mbf{E}^\dagger(\nu)\rangle$},
where $\mbf{E}^\dagger$ is the Hermitian transpose of $\mbf{E}$ and
the angular brackets denote time averaging.  More explicitly:
\begin{equation}
\bar{\bf{P}}(\nu)=
  \left( 
    \begin{array}{cc}
      \langle E_1(\nu)E_1^*(\nu)\rangle & \langle E_1(\nu)E_2^*(\nu)\rangle \\
      \langle E_2(\nu)E_1^*(\nu)\rangle & \langle E_2(\nu)E_2^*(\nu)\rangle
    \end{array}
  \right).
\label{eqn:avg_power_spectrum_matrix}
\end{equation}
Each component of the average power spectrum matrix,
$\bar{P}_{mn}(\nu)$, is the Fourier transform pair of the average
correlation function, $\bar{\rho}_{mn}(\tau)$ \citep{pap65}.
Therefore, $\bar{\bf{P}}(\nu)$ may be related to the commonly used
coherency matrix,
\begin{equation}
\boldsymbol{\rho}=
	\langle\mbf{E}(t)\boldsymbol{\otimes}\mbf{E}^\dagger(t)\rangle
=\bar{\boldsymbol{\rho}}(0)
={1\over2\pi}\int_{\nu_0-\Delta\nu}^{\nu_0+\Delta\nu}{\bar{\bf{P}}(\nu)}d\nu,
\label{eqn:coherency_matrix}
\end{equation}
where $\nu_0$ is the centre frequency and $2\Delta\nu$ is the
bandwidth of the observation.  The average power spectrum matrix may
therefore be interpreted as the coherency spectral density matrix and,
in the narrow band limit $\Delta\nu\rightarrow0$,
$\boldsymbol{\rho}=\bar{\bf{P}}(\nu_0)/2\pi$.

Using Equations~\ref{eqn:convolution_nu}
and~\ref{eqn:avg_power_spectrum_matrix} it is easily shown that a
two-dimensional linear system transforms the average power spectrum as
\begin{equation}
{\bar{\bf{P}}^\prime}(\nu)={\bf{J}}(\nu)\bar{\bf{P}}(\nu){\bf{J}}^\dagger(\nu).
\label{eqn:congruence_transformation}
\end{equation}
This matrix equation is a congruence transformation, and forms the
basis on which the frequency response of the system will be related to
the input (source) and output (measured) coherency spectrum.  For
brevity in this paper, all symbolic values are
assumed to be a function of frequency, $\nu$.

%%%%%%%%%%%%%%%%%%%%%%%%%%%%%%%%%%%%%%%%%%%%
%%%%%%%%%%%%%%%%%%%%%%%%%%%%%%%%%%%%%%%%%%%%
\section{Down-Conversion}
\label{app:downconversion}

By the Nyquist Theorem, a signal must be sampled at a rate equal to
twice its bandwidth in order to completely represent the information
discretely.  Therefore, subject to the finite recording rate of
digital observatory equipment, a radio astronomy experiment must be
constrained to observe a limited portion of the radio spectrum.  The
intermediate process by which the signal from the receiver is
band-limited and made ready for baseband recording is known as
down-conversion.

Consider the incoming radio signal, $x(t)$, and its Fourier transform,
$X(\nu)$, known as the spectrum, or spectral density of $x(t)$.  The
band-limited signal of interest, $x_b(t)$, is parameterized by its
centre frequency, $\nu_0$, and bandwidth, \bw.  Baseband
down-conversion is the process by which the spectral information
originally contained in the range [$\nu_0-\Delta\nu/2$,
$\nu_0+\Delta\nu/2$] is shifted down to [0, \bw].  It is worth noting
that this definition of baseband differs slightly from the more widely
accepted use in the telecommunications industry, where baseband refers
to the original band of frequencies of a signal before it is used to
modulate a carrier of much higher frequency.

% baseband 1. The original band of frequencies produced by a transducer,
% such as a microphone, telegraph key, or other signal-initiating
% device, prior to initial modulation. Note 1: In transmission systems,
% the baseband signal is usually used to modulate a carrier. Note 2:
% Demodulation re-creates the baseband signal. Note 3: Baseband
% describes the signal state prior to modulation, prior to multiplexing,
% following demultiplexing, and following demodulation. (188) Note 4:
% Baseband frequencies are usually characterized by being much lower in
% frequency than the frequencies that result when the baseband signal is
% used to modulate a carrier or subcarrier. 2. In facsimile, the
% frequency of a signal equal in bandwidth to that between zero
% frequency and maximum keying frequency. (188)

The spectral information is shifted to baseband by demodulating or
mixing the radio frequencies (RF) with a local oscillator (LO).  This
is equivalent to multiplying the signal, $x(t)$, with a pure tone,
$l(t)=a\cos(2\pi\nu_lt) + b\sin(2\pi\nu_lt)$.  By application of the
convolution theorem, and reference to \Tab{fourier}, mixing may also
be understood as a convolution with a pair of (complex) delta
functions in the frequency domain.  This understanding proves useful
in the following sections.

In addition to mixing to lower frequencies, the signal must also be
band-limited before analog-to-digital conversion.  Otherwise, power
from frequencies higher than the Nyquist frequency will be reflected
back into the band of interest, a pollution known as aliasing.  The
ideal low-pass filter is represented by the rectangle function (see
\Tab{fourier}) so that a bandpass filter with centre frequency,
$\nu_0$, and bandwidth, \bw, is given by
\begin{equation}
\Pi \left( {{|\nu|-\nu_0}\over\Delta\nu} \right).
\end{equation}
Note that the absolute value of $\nu$ in the first term of the above
equation creates a bandpass window at both positive and negative
frequency values.

Down-conversion therefore refers to the combined operation of mixing
and band-limiting.  The following sections describe in detail two
commonly used methods of down-conversion: dual-sideband (DSB) and
single-sideband (SSB).  These are also represented graphically in
Figures~\ref{fig:dsb} and~\ref{fig:ssb}.  The process of
down-conversion is performed separately and (ideally) identically on
each of the two orthogonal senses of polarization from the receiver
feed.

%%%%%%%%%%%%%%%%%%%%%%%%%%%%%%%%%%%%%%%%%%%%
%%%%%%%%%%%%%%%%%%%%%%%%%%%%%%%%%%%%%%%%%%%%
\subsection{Dual-Sideband Down-Conversion}
\label{app:dsb}

During dual-sideband down-conversion (DSB, see \Fig{dsb}), also known
as quadrature mixing, the voltages from the receiver are split equally
into two signal paths.  One signal is mixed with a local oscillator,
producing
\[
	i(t)=x(t)\cos(2\pi\nu_0t+\theta).
\]
The other signal is mixed with the same local oscillator phase-shifted
by 90 degrees,
\[
q(t)=x(t)\cos(2\pi\nu_0t+\theta-\pi/2)=x(t)\sin(2\pi\nu_0t+\theta).
\]
Both $i(t)$ and $q(t)$ are low-pass filtered with a cutoff frequency
of $\nu_c=\Delta\nu/2$, producing $i_b(t)=i(t)*\sinc(\pi\Delta\nu
t)$ and $q_b(t)=q(t)*\sinc(\pi\Delta\nu t)$.  The low-pass
filtered signals are then digitally sampled at the Nyquist rate of
$2\nu_c=\Delta\nu$.  The signals, $i_b(t)$ and $q_b(t)$ are known as
the in-phase and quadrature components, respectively, of $x(t)$ with
respect to $\nu_0$.

During playback, the analytic signal associated with $x(t)$ is formed
by taking:
\begin{eqnarray}
  z_b(t) & = & i_b(t)+\Ci q_b(t) \nonumber \\
       & = & [x(t)\cos(2\pi\nu_0 t)+\Ci x(t)\sin(2\pi\nu_0 t)] 
		* \sinc(\pi\Delta\nu t) \nonumber \\
       & = & [x(t)e^{2\pi\Ci\nu_0t}] * \sinc(\pi\Delta\nu t) \nonumber
\end{eqnarray}
where the distributive property of convolution has been applied and
the arbitrary phase angle, $\theta$, has been set to zero without loss
of generality.  In the Fourier domain,
\[
	Z_b(\nu)=X(\nu+\nu_0) \Pi(\nu/\Delta\nu).
\]
That is, the spectrum of $z_b(t)$ is equivalent to the band-limited
portion of $x(t)$ centred at $\nu_0$.  The negative frequency
components, centred at $-\nu_0$, have been suppressed by low-pass
filtering, forming the analytic signal associated with $x(t)$.

\begin{figure}[h]

\centerline{\includegraphics[height=10cm,angle=-90]{dsb.eps}}
%\centerline{\psfig{figure=dsb.ps,height=10cm,angle=-90}}
\caption{During dual-sideband down-conversion (DSB), the real
signal, X$(\nu)$, is split during mixing into its in-phase, I$(\nu)$,
and quadrature, Q$(\nu)$, components before low-pass filtering.  The
centre frequency and bandwidth of interest are $\nu_0$ and
\bw, respectively.  Each of the signals, I$_b(\nu)$ and Q$_b(\nu)$, 
have bandwidth $\Delta\nu/2$.  The complex signal,
Z$_b(\nu)$=I$_b(\nu)$+\Ci Q$_b(\nu)$, is the band-limited analytic
signal associated with X$(\nu)$.}
\label{fig:dsb}
\end{figure}

%%%%%%%%%%%%%%%%%%%%%%%%%%%%%%%%%%%%%%%%%%%%
%%%%%%%%%%%%%%%%%%%%%%%%%%%%%%%%%%%%%%%%%%%%
\subsection{Single-Sideband Down-Conversion}
\label{app:ssb}
%%%%%%%%%%%%%%%%%%%%%%%%%%%%%%%%%%%%%%%%%%%%
%%%%%%%%%%%%%%%%%%%%%%%%%%%%%%%%%%%%%%%%%%%%

During single-sideband down-conversion (SSB, see \Fig{ssb}), the
signal of interest, $x(t)$, is first bandpass filtered, producing
$x_b(t)$ where \mbox{$X_b(\nu)=X(\nu)\Pi((|\nu|-\nu_0)/\Delta\nu)$}.  The
band-limited signal is then mixed with a LO with frequency, $\nu_1$,
producing \mbox{$x_m(t)=x_b(t)\cos(2\pi\nu_1t+\theta)$}, where $\nu_1$
set to either $\nu_0+\Delta\nu/2$ (lower-sideband) or $\nu_0-\Delta\nu/2$
(upper-sideband).
After another stage of low-pass filtering, $x_u(t)$
is then digitally sampled at the Nyquist rate, $2\Delta\nu$.  As
bandpass filtering is performed before mixing, the filter used in the
first stage must be tunable over the range of interesting centre
frequencies, unlike the low-pass filter in a DSB down-converter.  For
this reason, DSB is often the more economical means of
down-conversion.

During playback, the analytic signal associated with $x(t)$ may be
formed in practice by taking the real-to-complex Fast Fourier
Transform (FFT), followed by the complex-to-complex inverse
FFT.  Most real-to-complex FFT implementations automatically omit the
redundant negative frequencies ($F(-\nu)=F^*(\nu)$) from their output,
implicitly producing the analytic signal (see
\App{quadrature_filter}).  Since many signal processing operations
(such as polyphase filter banks and phase-coherent dispersion removal)
are performed in the Fourier domain, the cost of calculating the
analytic signal is transparent.

\begin{figure}
\centerline{\includegraphics[height=5cm,angle=-90]{usb.eps}}
\caption{During single-sideband down-conversion (SSB), the real signal,
X$(\nu)$, is bandpass filtered before mixing.  The centre frequency
and bandwidth of interest are $\nu_0$ and \bw, respectively.
The band-limited signal, X$_b(\nu)$, may be mixed with a local
oscillator set to $\nu_0+\Delta\nu/2$ (lower-sideband) or
$\nu_0-\Delta\nu/2$ (upper-sideband, shown here). The resulting
signal, X$_m(\nu)$, is low-pass filtered, producing X$_u(\nu)$ with
bandwidth \bw.}
\label{fig:ssb}
\end{figure}


\end{appendix}

\bibliographystyle{aasjournal}
\bibliography{journals,modrefs,psrrefs,local,crossrefs}


\end{document}


